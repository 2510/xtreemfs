XtreemFS \cite{XtreemFS} is an object-based \cite{objStore,mesnier03objectbased} file system designed for Grid environments. It is the main distributed file system in the XtreemOS operating system, which relies on XtreemFS for replicated and low-latency file storage between Grid machines.

From a user's perspective, XtreemFS offers a global view on files. Files and directory trees are arranged into volumes. A volume can be mounted at any Grid node where a sufficiently authorized job can access and modify files on the volume. Applications access directories and files on XtreemFS volumes through normal POSIX\index{POSIX} interfaces (\texttt{open}, \texttt{read}, etc.) and thus do not require re-compilation in order to work with XtreemFS. This stands in marked contrast with earlier Grid file systems such as GFarm \cite{gfarm2}, which often forced users to rewrite parts of their applications in order to access files across the Grid via special non-POSIX\index{POSIX} APIs or to adapt to a non-POSIX\index{POSIX} file system semantics.

From an administrator's perspective, an XtreemFS installation consists of file system clients running on each user's machine and network-based services for storing and retrieving file metadata and data. The former services are known as Metadata and Replica Services (MRCs\index{MRC}), while the latter are called Object Storage Services (OSDs\index{OSD}). These services are complemented by the Replica Management Service (RMS), which is responsible for creating replicas on demand in response to changing user access patterns as well as eliminating redundant replicas; and the Object Sharing Service (OSS\index{OSS}), which provides transaction-based sharing of volatile memory objects and supports memory-mapped files for XtreemFS.

This deliverable is intended to serve as a developer guide. Its focus is on the current design and implementation of the XtreemFS client and servers, network protocols used between clients and servers, and test suites for XtreemFS.

\subsection{Document Structure}

The report is structured as follows. Sections \ref{sec:xtreemfs_mrc}, \ref{sec:xtreemfs_servers}, \ref{sec:xtreemfs_dir}, \ref{sec:xtreemfs_osd} describe the XtreemFS directory, metadata, and object store services. In section \ref{sec:xtreemfs_client} we introduce the new XtreemFS client, which was designed from the ground up to take advantage of the new binary protocol and to remedy numerous performance and scalability problems in the previous revision of the client. Section \ref{sec:xtreemfs_rms} concerns the XtreemFS Replica Management Service. We conclude with a discussion of recent testing efforts in section \ref{sec:xtreemfs_test}. Finally, section \ref{sec:xtreemfs_proto} documents the new binary client-server and server-server protocol, a more efficient and easily-maintained replacement for the text-based protocol of previous releases.
